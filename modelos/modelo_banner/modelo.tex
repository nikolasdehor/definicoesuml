\documentclass{modelo_banner}

\primeiroAutor{Maria Torres}
\abreviacaoPrimeiroAutor{TORRES, M.}
\notaPrimeiroAutor{Estudante do curso de bacharelado em Sistemas de Informação, IF Goiano – Campus Ceres.}

\segundoAutor{Bob Camargo Silva}
\abreviacaoSegundoAutor{SILVA, B. C.}
\notaSegundoAutor{Estudante do curso Técnico em Meio Ambiente, IF Goiano – Campus Ceres.}

\terceiroAutor{Ana Maria Cunha}
\abreviacaoTerceiroAutor{CUNHA, A. M.}
\notaTerceiroAutor{Estudante do curso Técnico em Informática, IF Goiano – Campus Ceres.}

\quartoAutor{Pedro Alvares Cabral}
\abreviacaoQuartoAutor{CABRAL, P. A.}
\notaReferenciaQuartoAutor{3}

\quintoAutor{Joana da Silva Amaral}
\abreviacaoQuintoAutor{AMARAL, J. S.}
\notaReferenciaQuintoAutor{2}

\sextoAutor{Ronneesley Moura Teles}
\abreviacaoSextoAutor{TELES, R. M.}
\notaSextoAutor{Professor orientador, IF Goiano – Campus Ceres.}

\titulo{Ex.: Comparação do desempenho das linguagens Julia e Java}

\begin{document}

	\construirtitulo

	\construirautores

	\begin{multicols}{2}	
		\section{Introdução}
		
		Apresentação do problema, onde ele acontece e sua importância.
		
		Apresentação da literatura envolvida no estudo, ou seja, revisão bibliográfica do problema realçando sua importância.
		
		Sugere-se no último parágrafo deste item incluir objetivo ou objetivos com o trabalho \cite{texbook}.
		
		\section{Materiais e Métodos}
		
		Esta seção também pode ser chamada de: ``Desenvolvimento''.
		
		Apresentar como os dados foram coletados do experimento \cite{knuth:1984}.
		
		Apresentar a forma estatística empregada no problema, ou o método para resolução do problema.
		
		Destacar quais linguagens e ferramentas foram usadas e suas respectivas versões.
		
		Mostrar fotos do experimento caso tenha. Ex.: foi feita uma máquina para coletar a luz que ultrapassa o vidro, então mostre a máquina feita em uma foto.
		
		Pode-se colocar um fluxograma simplificado de como o trabalho foi feito \cite{lesk:1977}.
		
		\section{Resultados}
		
		Apresentar em forma de gráficos, principalmente, os resultados obtidos no estudo.
		
		Este também é o momento para discussão do que foi observado, geralmente esta seção se chama: ``Resultado e discussão''.
		
		Embasar essa discussão com referências e citações geralmente é muito bom.
		
		\section{Considerações finais}
		
		Pontuar as principais conclusões do trabalho. 
		
		Colocá-las em forma de tópicos, geralmente é muito bem visto. Ex.:
		
		\begin{itemize}
			\item A linguagem Julia mostrou-se $X$ vezes mais rápida que Python nos experimentos;
			
			\item Foi possível otimizar os recursos da empresa diminuindo os gastos em 30\%;
		\end{itemize}
		
		\section{Agradecimentos}
		
		Nesta seção são feitos os agradecimentos a quem considerar importante.
		
		\referencias		
		
	\end{multicols}
	
	\rodape

\end{document}